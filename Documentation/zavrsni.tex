\documentclass[times, utf8, zavrsni, numeric]{fer}
\usepackage{booktabs}

\begin{document}

\thesisnumber{5187}
\title{Ekstrakcija tablica na skeniranim dokumentima}
\author{Kristijan Vulinović}

\maketitle

% Ispis stranice s napomenom o umetanju izvornika rada. Uklonite naredbu \izvornik ako želite izbaciti tu stranicu.
\izvornik

% Dodavanje zahvale ili prazne stranice. Ako ne želite dodati zahvalu, naredbu ostavite radi prazne stranice.
\zahvala{Zahvala... to be written...}

\tableofcontents

\chapter{Uvod}
U današnje vrijeme postoje izuzetno velike količine papirnatih dokumenata.
Samo u Sjedinjenim Američkim Država nastaje više od milijarde novih papirnatih dokumenata svakog radnog dana. 
Mogućnost digitalizacije takvih dokumenata može biti od velike koristi prilikom pohrane, slanja ili pretraživanja istih. \cite{article:Skew-detection}


\chapter{Binarizacija slike}
Knjiga \cite{book:Two-Dimensional-Signal-Image-Processing} opisuje kako koristiti adaptivni wiener filter prema nekim tamo formulama koje su krive u drugim mjestima. 

\chapter{Zaključak}
Zaključak.


\bibliography{literatura}
\bibliographystyle{fer}

\begin{sazetak}
Sažetak na hrvatskom jeziku.

\kljucnerijeci{Ključne riječi, odvojene zarezima.}
\end{sazetak}

\engtitle{Table Extraction on Scanned Documents}
\begin{abstract}
Abstract.

\keywords{Keywords.}
\end{abstract}

\end{document}
