\documentclass[times, utf8, zavrsni, numeric]{fer}
\usepackage{booktabs}

\begin{document}

\thesisnumber{5187}
\title{Ekstrakcija tablica na skeniranim dokumentima}
\author{Kristijan Vulinović}

\maketitle

% Ispis stranice s napomenom o umetanju izvornika rada. Uklonite naredbu \izvornik ako želite izbaciti tu stranicu.
\izvornik

% Dodavanje zahvale ili prazne stranice. Ako ne želite dodati zahvalu, naredbu ostavite radi prazne stranice.
\zahvala{Zahvala... to be written...}

\tableofcontents

\chapter{Uvod}
U današnje vrijeme postoje izuzetno velike količine papirnatih dokumenata.
Samo u Sjedinjenim Američkim Državama nastaje više od milijarde novih papirnatih dokumenata svakog radnog dana. 
Mogućnost digitalizacije takvih dokumenata može biti od velike koristi prilikom pohrane, slanja ili pretraživanja istih. \cite{article:Skew-detection}
Digitalizaciju dokumenata možemo podijeliti u dva dijela: prepoznavanje teksta te prepoznavanje grafičkih objekata. \cite{conference:DetectionOfTableStructure} 
Za prepoznavanje teksta je dostupan velik broj alata koji omogućuju optičko prepoznavanje znakova (engl. \textit{optical character recognition}).
Prepoznavanje grafičkih objekata dokumenta mnogo je manje zastupljeno u odnosu na prepoznavanje teksta.
U to se svrstavaju linije, oblici, slike, simboli, tablice i razni drugi objekti koji se mogu nalaziti na skeniranim dokumentima.
Ovaj rad se fokusira isključivo na prepoznavanje tablica, što je prethodno već opisano u radovima poput \cite{conference:DetectionOfTableStructure} i \cite{conference:AutomaticTableDetectionInDocumentImages}. 
Taj postupak se dijeli na prepoznavanje položaja tablice u odnosu na ostatak dokumenta, prilikom čega je potrebno u dokumentu izdvojiti tablicu od ostatka teksta i ostalih grafičkih objekata, a što je opisano u radu \cite{article:Medium-IndependentTableDetection}.
Nakon što je tablica pronađena određuje se njezin izgled, odnosno broj redaka i stupaca, odnosno koordinate pojedine ćelije, a što je detaljnije opisano u nastavku rada. \\

Predstavljeno rješenje počinje od slike u sivim tonovima (engl. \textit{gray-scale}), koja se binarizira kako bi se dobila slika koja se sastoji od isključivo crne i bijele boje.
Dobivena crno-bijela slika koristi se u daljnjoj obradi te se provjerava je li slika rotirana, odnosno kut rotacije iste, nakon čega se rotacija ispravlja.
Ovako obrađena slika koristi se dalje za detekciju tablica, \textbf{što će biti objašnjeno jednom kada odlučim kako to radim.}


\chapter{Binarizacija slike}
Knjiga \cite{book:Two-Dimensional-Signal-Image-Processing} opisuje kako koristiti adaptivni wiener filter prema nekim tamo formulama koje su krive u drugim mjestima. 

\chapter{Zaključak}
Zaključak.


\bibliography{literatura}
\bibliographystyle{fer}

\begin{sazetak}
Sažetak na hrvatskom jeziku.

\kljucnerijeci{Ključne riječi, odvojene zarezima.}
\end{sazetak}

\engtitle{Table Extraction on Scanned Documents}
\begin{abstract}
Abstract.

\keywords{Keywords.}
\end{abstract}

\end{document}
